% Report.
%
% Copyright (C) 2010, 2011  Vladimir Rutsky <altsysrq@gmail.com>
%
% This work is licensed under a Creative Commons Attribution-ShareAlike 3.0 
% Unported License. See <http://creativecommons.org/licenses/by-sa/3.0/> 
% for details.

% TODO: Use styles according to GOST (it's hard).
% TODO: Made legend simpler to understand.
% TODO: Provide in text exact parameters of tested models (window size,
% loss probability, etc).

\documentclass[a4paper,10pt]{article}

% Encoding support.
\usepackage{ucs}
\usepackage[utf8x]{inputenc}
\usepackage[T2A]{fontenc}
\usepackage[russian]{babel}

\usepackage{amsmath, amsthm, amssymb}

% Indenting first paragraph.
\usepackage{indentfirst}

\usepackage{url}
\usepackage[unicode]{hyperref}

%\usepackage[final]{pdfpages}

\usepackage[pdftex]{graphicx}
\usepackage{subfig}

\usepackage{fancyvrb}
\usepackage{color}
\usepackage{texments}
%\usepackage{listings}

\newcommand{\HRule}{\rule{\linewidth}{0.5mm}}

% Spaces after commas.
\frenchspacing
% Minimal carrying number of characters,
\righthyphenmin=2

% From K.V.Voroncov Latex in samples, 2005.
\textheight=24cm   % text height
\textwidth=16cm    % text width.
\oddsidemargin=0pt % left side indention
\topmargin=-1.5cm  % top side indention.
\parindent=24pt    % paragraph indent
\parskip=0pt       % distance between paragraphs.
\tolerance=2000
%\flushbottom       % page height aligning
%\hoffset=0cm
%\pagestyle{empty}  % without numeration

\newcommand{\myemail}[1]{%
\href{mailto:#1}{\nolinkurl{#1}}}

\newcommand{\myfunc}[1]{%
\textit{#1}}

\begin{document}

% Title page.
% title.tex
% Report title page.
% Copyright (C) 2010  Vladimir Rutsky <altsysrq@gmail.com>

\begin{titlepage} % начало титульной страницы

\begin{center} % включить выравнивание по центру

\large Санкт-Петербургский государственный политехнический университет\\[5.5cm]
% название института, затем отступ 5,5см

\huge Отчет по курсовой работе\\[0.6cm] % название работы, затем отступ 0,6см
\large по~курсу <<Верификация программ>>\\[1cm]
\large <<Разработка контроллера светофоров на перекрёстке и его верификация>>\\[6cm]
% тема работы, затем отступ 6см

\begin{flushright} % выровнять её содержимое по правому краю
\begin{tabular}{l l}
Студент: & Руцкий~В.\,В.\\
Группа: & 5057/2\\
Вариант: & 9, 13\\
Преподаватель: & Шошмина~И.\,В.
\end{tabular}
\end{flushright} % конец выравнивания по правому краю

\vfill % заполнить всё доступное ниже пространство

{\large Санкт-Петербург 2010}
\end{center} % закончить выравнивание по центру
\thispagestyle{empty} % не нумеровать страницу
\end{titlepage} % конец титульной страницы


\tableofcontents
\pagebreak

% Content

\section{Постановка задачи}
Необходимо реализовать модель канального уровня сетевой модели OSI
и исследовать эффективность используемого в модели протокола передачи данных 
при наличии ошибок при передаче.

Для модели необходимо использовать протокол, 
использующий идею плавающего окна.

Передача ведётся по полнодуплексному каналу.
Данные должны передаваться с гарантией доставки.

\section{Выбранный метод решения}
\label{section:solution}
Для реализации был выбран протокол выборочного повтора (\textit{selective repeat}), 
описанный в~\cite{tanenbaum2003compnet}.

Протокол использует идею плавающего окна:
отправитель отправляет кадры в пределах некоторого <<окна>>~--- 
интервал ограниченной длины последовательно идущих кадров;
получатель ожидает кадров также в пределах своего рабочего окна.

При получении кадра получатель отправляет уведомление о доставке 
отправителю.
При получении уведомления о доставке отправитель сдвигает рабочее окно
таким образом, чтобы для первого кадра окна ещё не было получено
уведомления о доставке.
Аналогичным образом получатель поддерживает своё рабочее окно.

Для каждого отправленного кадра отправитель создаёт таймер, 
который подсчитывает сколько времени кадр находится в состоянии ожидания
уведомления о доставке.
В случае превышения времени ожидания для кадра устанавливается флаг ошибки доставки.

При обнаружении кадра с ошибкой доставки производится его повторная отправка.

\section{Детали реализации}
\subsection{Физический уровень передачи данных}
Модель физического уровня передачи данных представлена 
классом узла \myfunc{FullDuplexNode}
(см.~приложение~\ref{appendix:sources:duplex-link}).
Узлы конструируются связанными парами.
Каждый узел предоставляет два метода: \myfunc{write} и \myfunc{read}, 
позволяющие отправлять связанному узлу непрерывный поток байтов.

В реализации физического уровня предусмотрено внесение ошибок в передаваемые данные.
Возможно внесение ошибок трёх типов: подмена передаваемого байта, 
добавление лишнего байта, удаление передаваемого байта.

\subsection{Передача данных низкоуровневыми кадрами}
Для удобства контроля передачи данных 
непрерывный поток байтов делится на группы байтов переменной длины~--- 
низкоуровневые кадры
(см.~приложение~\ref{appendix:sources:frame}).
Деление на низкоуровневые кадры делается по аналогии с делением в протоколе SLIP.%
\footnote{\textit{Serial Line Internet Protocol}, описан в RFC 1055.}

\subsection{Передача данных кадрами с гарантией доставки}
Поверх уровня передачи низкоуровневыми кадрами 
реализован протокол передачи кадров с выборочным повтором,
описанный в разделе~\ref{section:solution} 
(см.~реализацию в приложении~\ref{appendix:sources:sliding-window}).

Передаваемые данные делятся на последовательные блоки небольшого размера и 
упаковываются в кадры,
далее кадры передаются смежному узлу через 
нижележащий уровень сети.

Используется кадр следующего формата (см.~таблицу~\ref{table:frame}):
\begin{table}[h]
  \begin{center}
    \begin{tabular}{|c|c|c|c|c|c|}
      \hline
      type & id & last & len & data & CRC32 \\
      \hline
    \end{tabular}
  \end{center}
  \caption{Формат кадра.}
  \label{table:frame}
\end{table}
\begin{itemize}
  \item \textbf{type} (1 байт)~--- 
  тип кадра: данные или подтверждение полученных данных;
  \item \textbf{id} (2 байта)~--- 
  идентификатор (номер) передаваемого кадра;
  \item \textbf{last} (1 байт)~--- 
  является ли кадр последней частью передаваемых данных;
  \item \textbf{len} (4 байта)~--- длина поля данных;
  \item \textbf{data} (\textbf{len} байт)~--- данные;
  \item \textbf{CRC32} (4 байта)~--- контрольная сумма пакета (CRC32%
\footnote{\textit{Сyclic redundancy check}~--- \textit{циклический избыточный код}, %
описан в~\cite{peterson1961crc}.%
}).
\end{itemize}

\section{Исследования алгоритма}
В результате проведённых исследований были получены следующие результаты.

На рис.~\ref{fig:graph:wsize} и рис.~\ref{fig:graph:wsize-loss}
представлены полученные зависимости времени передачи сообщений 
и количества переданных кадров от размера окна передачи.
\begin{figure}
  \centering
  \subfloat[Передача без потерь.]%
{\label{fig:graph:wsize}\includegraphics[height=5cm]{graph_wsize.pdf}}
  \quad
  \subfloat[Передача с потерями.]%
{\label{fig:graph:wsize-loss}\includegraphics[height=5cm]{graph_wsize_loss.pdf}}
  \caption{Зависимость времени передачи данных и количества переданных низкоуровневых кадров от размера окна передачи.}
  \label{fig:graph:wsize-all}
\end{figure}

На рис.~\ref{fig:graph:maxframe} и рис.~\ref{fig:graph:maxframe-loss}
представлены полученные зависимости времени передачи сообщений 
и количества переданных кадров от максимального размера низкоуровнего кадра.
\begin{figure}
  \centering
  \subfloat[Передача без потерь.]%
{\label{fig:graph:maxframe}\includegraphics[height=5cm]{graph_maxframe.pdf}}
  \quad
  \subfloat[Передача с потерями.]%
{\label{fig:graph:maxframe-loss}\includegraphics[height=5cm]{graph_maxframe_loss.pdf}}
  \caption{Зависимость времени передачи данных и количества переданных низкоуровневых кадров 
  от максимального размера низкоуровнего кадра.}
  \label{fig:graph:maxframe-all}
\end{figure}

На рис.~\ref{fig:graph:loss}
представлена полученная зависимость времени передачи сообщений 
и количества переданных кадров от вероятности потери данных при передаче.
\begin{figure}
  \centering
  \includegraphics[height=5cm]{graph_loss.pdf}
  \caption{Зависимость времени передачи данных и количества переданных низкоуровневых кадров 
  от вероятности потери данных.}
  \label{fig:graph:loss}
\end{figure}

\pagebreak
\section{Результат работы}
Исследование поведения реализованной модели в различных условиях показало:
\begin{itemize}
  \item Малые размеры окна передачи сильно замедляют работу сети.
  При увеличении размера окна, 
  с некоторого момента увеличение не улучшает скорость работы сети.
  \item В случае наличия потерь при передаче, 
  при увеличении размера кадра увеличивается процент недоставленных кадров,
  что замедляет работу сети. 
  При уменьшении размера кадра процент недоставленных кадров уменьшается,
  но растёт общее количество передаваемых кадров, что с некоторого момента
  также сильно замедляет работу сети.
  \item Время передачи данных и количество посылаемых низкоуровневых кадров 
  экспоненциально зависит от вероятности потери данных.
\end{itemize}

\pagebreak

\appendix
\section{Исходный код}
\label{appendix:sources}

\usestyle{default}

\subsection{Полнодуплексная передача данных}
\label{appendix:sources:duplex-link}
\includecode[python -O linenos=1]{data/duplex_link.py}

\subsection{Передача данных низкоуровневыми кадрами}
\label{appendix:sources:frame}
\includecode[python -O linenos=1]{data/frame.py}

\subsection{Передача данных кадрами с гарантией доставки}
\label{appendix:sources:sliding-window}
\includecode[python -O linenos=1]{data/sliding_window.py}

% TODO: Add example of log.

\pagebreak

\bibliographystyle{unsrt}
\bibliography{references}

\end{document}
