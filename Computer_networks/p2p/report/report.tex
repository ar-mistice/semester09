% Report.
%
% Copyright (C) 2010  Vladimir Rutsky <altsysrq@gmail.com>
%
% This work is licensed under a Creative Commons Attribution-ShareAlike 3.0 
% Unported License. See <http://creativecommons.org/licenses/by-sa/3.0/> 
% for details.

% TODO: Use styles according to GOST (it's hard).

\documentclass[a4paper,10pt]{article}

% Encoding support.
\usepackage{ucs}
\usepackage[utf8x]{inputenc}
\usepackage[T2A]{fontenc}
\usepackage[russian]{babel}

\usepackage{amsmath, amsthm, amssymb}

% Indenting first paragraph.
\usepackage{indentfirst}

\usepackage{url}
\usepackage[unicode]{hyperref}

%\usepackage[final]{pdfpages}

\usepackage[pdftex]{graphicx}
\usepackage{subfig}

%TODO: use texments
\usepackage{fancyvrb}
\usepackage{color}
\usepackage{texments}
%\usepackage{listings}

\newcommand{\HRule}{\rule{\linewidth}{0.5mm}}

% Spaces after commas.
\frenchspacing
% Minimal carrying number of characters,
\righthyphenmin=2

% From K.V.Voroncov Latex in samples, 2005.
\textheight=24cm   % text height
\textwidth=16cm    % text width.
\oddsidemargin=0pt % left side indention
\topmargin=-1.5cm  % top side indention.
\parindent=24pt    % paragraph indent
\parskip=0pt       % distance between paragraphs.
\tolerance=2000
%\flushbottom       % page height aligning
%\hoffset=0cm
%\pagestyle{empty}  % without numeration

\newcommand{\myemail}[1]{%
\href{mailto:#1}{\nolinkurl{#1}}}

\begin{document}

% Title page.
% title.tex
% Report title page.
% Copyright (C) 2010  Vladimir Rutsky <altsysrq@gmail.com>

\begin{titlepage} % начало титульной страницы

\begin{center} % включить выравнивание по центру

\large Санкт-Петербургский государственный политехнический университет\\[5.5cm]
% название института, затем отступ 5,5см

\huge Отчет по курсовой работе\\[0.6cm] % название работы, затем отступ 0,6см
\large по~курсу <<Верификация программ>>\\[1cm]
\large <<Разработка контроллера светофоров на перекрёстке и его верификация>>\\[6cm]
% тема работы, затем отступ 6см

\begin{flushright} % выровнять её содержимое по правому краю
\begin{tabular}{l l}
Студент: & Руцкий~В.\,В.\\
Группа: & 5057/2\\
Вариант: & 9, 13\\
Преподаватель: & Шошмина~И.\,В.
\end{tabular}
\end{flushright} % конец выравнивания по правому краю

\vfill % заполнить всё доступное ниже пространство

{\large Санкт-Петербург 2010}
\end{center} % закончить выравнивание по центру
\thispagestyle{empty} % не нумеровать страницу
\end{titlepage} % конец титульной страницы


\tableofcontents
\pagebreak

% Content

\section{Постановка задачи}
Необходимо реализовать модели канального уровня сетевой модели OSI, 
использующие различные протоколы передачи данных peer-to-peer, 
и сравнить эффективность их работы при передаче данных с ошибками.

Для сравнения необходимо взять алгоритмы, 
использующие идею плавающего окна.

Передача ведётся по полнодуплексному каналу.
Данные должны передаваться с гарантией доставки.

\section{Выбранный метод решения}
Для сравнения были выбраны следующие протоколы, описанные в~\cite{tanenbaum2003compnet}:
\begin{itemize}
  \item протокол с возвратом на $n$ (\textit{go back $n$}),
  \item протокол с выборочным повтором (\textit{selective repeat}).
\end{itemize}

Оба протокола используют идею плавающего окна:
отправитель отправляет кадры в пределах некоторого <<окна>>~--- 
интервал ограниченной длины последовательно идущих кадров;
получатель ожидает кадров также в пределах своего рабочего окна.

При получении кадра получатель отправляет уведомление о доставке 
отправителю.
При получении уведомления о доставке отправитель сдвигает рабочее окно
таким образом, чтобы для первого кадра окна ещё не было получено
уведомления о доставке.
Аналогичным образом получатель поддерживает своё рабочее окно.

Для каждого отправленного кадра отправитель создаёт таймер, 
который подсчитывает сколько времени кадр находится в состоянии ожидания
уведомления о доставке.
В случае превышения времени ожидания для кадра устанавливается флаг ошибки доставки.

Рассматриваемые протоколы отличаются стратегией обработки недоставленных кадров.

\subsection{Протокол с возвратом на $n$}
В протоколе с возвратом на $n$ обработка кадров с ошибкой доставки происходит следующим образом:
при обнаружении первого кадра с ошибкой доставки, обозначим его номер как $n$,
производится повторная посылка всех кадров начиная с $n$.

\subsection{Протокол с выборочным повтором}
В протоколе с выборочным повтором обработка кадров с ошибкой доставки происходит следующим образом:
при обнаружении кадра с ошибкой доставки производится его повторная отправка.

\section{Результат работы}

\pagebreak

\appendix
\section{Исходный код}
\label{appendix-sources}

%\lstinputlisting[language=Python,numbers=left,%
%caption={Описание классов полнодуплексного соединения},%
%label=src-duplex-link]{data/duplex_link.py}

\usestyle{default}
\includecode{data/duplex_link.py}

\pagebreak

\bibliographystyle{unsrt}
\bibliography{references}

\end{document}
