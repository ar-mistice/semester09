% Research work task for 9th semester in SPbSTU.
% Copyright (C) 2010 Vladimir Rutsky <altsysrq@gmail.com>

\documentclass[a4paper,10pt]{article}

% Encoding support.
\usepackage{ucs}
\usepackage[utf8x]{inputenc}
\usepackage[T2A]{fontenc}
\usepackage[russian,english]{babel}

\usepackage{amsmath, amsthm, amssymb}

% Spaces after commas.
\frenchspacing
% Minimal carrying number of characters,
\righthyphenmin=2

% From K.V.Voroncov Latex in samples, 2005.
\textheight=24cm   % text height
\textwidth=16cm    % text width.
\oddsidemargin=0pt % left side indention
\topmargin=-1.5cm  % top side indention.
\parindent=24pt    % paragraph indent
\parskip=0pt       % distance between paragraphs.
\tolerance=2000
%\flushbottom       % page height aligning
%\hoffset=0cm
\pagestyle{empty}  % without numeration

% Indenting first paragraph.
\usepackage{indentfirst}

\usepackage{setspace}
%\linespread{1.5}

\usepackage{enumitem}
%\usepackage{datetime}

% from http://www.latex-community.org/forum/viewtopic.php?f=5&t=1383&start=0&st=0&sk=t&sd=a
\makeatletter
\renewcommand\paragraph{\@startsection{paragraph}{4}{\z@}%
  {-3.25ex\@plus -1ex \@minus -.2ex}%
  {1.5ex \@plus .2ex}%
  {\normalfont\normalsize\bfseries}}
\makeatother

\begin{document}
\selectlanguage{russian}

\begin{center}
\begin{spacing}{1.3}
  {\Large\bfseries Задание на НИРС} \\
  {\large Руцкий Владимир Владимирович, гр.~5057/2, 9~семестр 2010~г.}
\end{spacing}
\end{center}

\noindent\textbf{Тема работы:}\quad Решение задачи построения ортофотоплана по аэрофотоснимкам с ярко выраженным параллаксом \\
\textbf{Место выполнения:}\quad ЗАО <<Транзас Новые Технологии>> \\
\textbf{Руководитель:}\quad Ковалёв Антон Сергеевич, 
магистр прикладной математики и информатики, 
ведущий инженер-программист (руководитель группы) ЗАО <<Транзас Новые Технологии>>

\paragraph{Постановка задачи}
Даны два аэрофотоснимка поверхности Земли, сделанных одной камерой из двух разных точек.

Высота съёмки над уровнем Земли находится в диапазоне от 800 до 1600 метров.
При съёмке с такой высоты видимая поверхность Земли считается плоскотью с направленной вверх нормалью 
(нормаль плоскости земли дана, плоскость~--- нет).
Даны углы зрения камеры, которые лежат в диапазоне от $30^{\circ}$ до $60^{\circ}$.
Для каждого снимка задана ориентация камеры в момент съёмки.
Направление съёмки отклоняется от вертикального вниз не более чем на $30^{\circ}$.

Положения камеры в пространстве в момент съёмки не даны, но
дано, что области поверхности Земли, запечатлённые на снимках, пересекаются не менее чем половиной своей площади,
и объекты, стоящие на снимаемой поверхности и запечатлённые на обоих снимках, видны под разными углами 
(угол, между направлениями из объекта в точки положения камеры в момент съёмки, 
лежит в диапазоне от $30^{\circ}$ до $60^{\circ}$).

На снимаемой поверхности Земли встречаются объекты возвышающиеся над её плоскостью на высоту до 40 метров: 
строения, деревья; 
и объекты опускающиеся ниже уровня плоскости Земли на глубину до нескольких метров: канавы, рвы.
В результате съёмки таких объектов из разных точек возникает являние параллакса:
возвышающиеся объекты на разных снимках загораживают разную часть снимаемой поверхности.

Требуется построить ортофотоплан снятой из двух точек поверхности Земли.

\paragraph{Элементы исследования и новизны}
Определение плоскости поверхности Земли, 
необходимой для построения преобразований, требуемых для получения ортофотоплана,
является нетривиальной задачей в данных условиях.

\paragraph{Ожидаемый результат}
Решение поставленной задачи позволит улучшить качество ортофотопланов, 
получаемых из аэрофотоснимков, сделанных на низких высотах. 

\paragraph{Среда разработки}

Платформа: Microsoft Windows XP.

Языки программирования: C++, Python.

Среда разработки: Microsoft Visual Studio, NetBeans.

\paragraph{Литература}

\begin{enumerate}[itemsep=\itemsep]
  \item R.\,Szeliski.\quad Image Alignment and Stitching: A Tutorial.\quad Now Publishers Inc, 2006.
  \item R.\,Szeliski.\quad Computer Vision: Algorithms and Applications.\quad Springer, 2010.
\end{enumerate}
%\vspace{0cm}

%\begin{flushbottom}
\begin{flushright}
\begin{spacing}{2}
\begin{tabular}{l l l}
Студент & \noindent\underline{\makebox[2.5cm][l]{}} & В.\,В.\,Руцкий \\
Руководитель & \noindent\underline{\makebox[2.5cm][l]{}} & А.\,С.\,Ковалёв\\
\end{tabular}
\vspace{0.3cm}

\today
\end{spacing}
\end{flushright}
%\end{flushbottom}
\end{document}
