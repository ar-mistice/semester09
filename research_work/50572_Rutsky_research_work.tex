% Research work task for 9th semester in SPbSTU.
% Copyright (C) 2010 Vladimir Rutsky <altsysrq@gmail.com>

\documentclass[a4paper,10pt]{article}

% Encoding support.
\usepackage{ucs}
\usepackage[utf8x]{inputenc}
\usepackage[T2A]{fontenc}
\usepackage[russian]{babel}

\usepackage{amsmath, amsthm, amssymb}

% Spaces after commas.
\frenchspacing
% Minimal carrying number of characters,
\righthyphenmin=2

% From K.V.Voroncov Latex in samples, 2005.
\textheight=24cm   % text height
\textwidth=16cm    % text width.
\oddsidemargin=0pt % left side indention
\topmargin=-1.5cm  % top side indention.
\parindent=24pt    % paragraph indent
\parskip=0pt       % distance between paragraphs.
\tolerance=2000
%\flushbottom       % page height aligning
%\hoffset=0cm
\pagestyle{empty}  % without numeration

% Indenting first paragraph.
\usepackage{indentfirst}

\usepackage{setspace}
%\linespread{1.5}

% from http://www.latex-community.org/forum/viewtopic.php?f=5&t=1383&start=0&st=0&sk=t&sd=a
\makeatletter
\renewcommand\paragraph{\@startsection{paragraph}{4}{\z@}%
  {-3.25ex\@plus -1ex \@minus -.2ex}%
  {1.5ex \@plus .2ex}%
  {\normalfont\normalsize\bfseries}}
\makeatother

\begin{document}

\begin{center}
\begin{spacing}{1.5}
  {\Large\bfseries Задание на НИРС} \\
  {\large Руцкий Владимир Владимирович, гр.~5057/2, 9~семестр 2010~г.}
\end{spacing}
\end{center}

\noindent\textbf{Тема работы:}\quad Разработка технологии создания ортофотоплана местности \\
\textbf{Место выполнения:}\quad ЗАО <<Транзас Новые Технологии>> \\
\textbf{Руководитель:}\quad Ковалёв Антон Сергеевич, 
магистр прикладной математики и информатики, 
ведущий инженер-программист (руководитель группы) ЗАО <<Транзас Новые Технологии>>

\paragraph{Постановка задачи}

Test

\paragraph{Элементы исследования и новизны}

Тест

\paragraph{Ожидаемый результат}

Тест

\paragraph{Среда разработки}

Платформа: Microsoft Windows XP

Языки программирования: C++, Python

Среда разработки: Microsoft Visual Studio, NetBeans

\paragraph{Литература}

\begin{enumerate}
  \item R.\,Szeliski.\quad Image Alignment and Stitching: A Tutorial.\quad Now Publishers Inc, 2006.
  \item R.\,Szeliski.\quad Computer Vision: Algorithms and Applications.\quad Springer, 2010.
\end{enumerate}

\vspace{3cm}

\begin{flushright} % выровнять её содержимое по правому краю
\begin{spacing}{2}

\begin{tabular}{l l l}
Студент & \noindent\underline{\makebox[2.5cm][l]{}} & В.\,В.\,Руцкий \\
Руководитель & \noindent\underline{\makebox[2.5cm][l]{}} & А.\,С.\,Ковалёв\\
\end{tabular}

\vspace{1cm}
29~сентября~2010~г.

\end{spacing}
\end{flushright} % конец выравнивания по правому краю

\end{document}
